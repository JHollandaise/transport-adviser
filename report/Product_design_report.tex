%%%%%%%%%%%%%%%%%%%%%%%%%%%%%%%%%%%%%%%%%
% Short Sectioned Assignment
% LaTeX Template
% Version 1.0 (5/5/12)
%
% This template has been downloaded from:
% http://www.LaTeXTemplates.com
%
% Original author:
% Frits Wenneker (http://www.howtotex.com)
%
% License:
% CC BY-NC-SA 3.0 (http://creativecommons.org/licenses/by-nc-sa/3.0/)
%
%%%%%%%%%%%%%%%%%%%%%%%%%%%%%%%%%%%%%%%%%

%----------------------------------------------------------------------------------------
%	PACKAGES AND OTHER DOCUMENT CONFIGURATIONS
%----------------------------------------------------------------------------------------

\documentclass[paper=a4, fontsize=12pt]{scrartcl} % A4 paper and 11pt font size

\usepackage[T1]{fontenc} % Use 8-bit encoding that has 256 glyphs
\usepackage{fourier} % Use the Adobe Utopia font for the document - comment this line to return to the LaTeX default
\usepackage[english]{babel} % English language/hyphenation
\usepackage{amsmath,amsfonts,amsthm,amssymb,gensymb} % Math packages

\usepackage{graphicx}
\graphicspath{{/Users/josephholland/Documents/transport_adviser/report/images/}}
\usepackage{caption}
\usepackage{subcaption}

\usepackage[none]{hyphenat}

\usepackage{lscape}
\usepackage{array}

\usepackage{tikz,calc}
\usepackage{forest}
\usetikzlibrary{arrows.meta}


\usepackage{sectsty} % Allows customizing section commands
\allsectionsfont{\centering \normalfont\scshape} % Make all sections centered, the default font and small caps

\usepackage{fancyhdr} % Custom headers and footers
\pagestyle{fancyplain} % Makes all pages in the document conform to the custom headers and footers
\fancyhead{} % No page header - if you want one, create it in the same way as the footers below
\fancyfoot[L]{} % Empty left footer
\fancyfoot[C]{} % Empty center footer
\fancyfoot[R]{\thepage} % Page numbering for right footer
\renewcommand{\headrulewidth}{0pt} % Remove header underlines
\renewcommand{\footrulewidth}{0pt} % Remove footer underlines
\setlength{\headheight}{13.6pt} % Customize the height of the header

\numberwithin{equation}{section} % Number equations within sections (i.e. 1.1, 1.2, 2.1, 2.2 instead of 1, 2, 3, 4)
\numberwithin{figure}{section} % Number figures within sections (i.e. 1.1, 1.2, 2.1, 2.2 instead of 1, 2, 3, 4)
\numberwithin{table}{section} % Number tables within sections (i.e. 1.1, 1.2, 2.1, 2.2 instead of 1, 2, 3, 4)

\setlength\parindent{0pt} % Removes all indentation from paragraphs - comment this line for an assignment with lots of text

%----------------------------------------------------------------------------------------
%	TITLE SECTION
%----------------------------------------------------------------------------------------

\newcommand{\horrule}[1]{\rule{\linewidth}{#1}} % Create horizontal rule command with 1 argument of height

\title{
\normalfont \normalsize
\textsc{University of Cambridge} \\ [25pt] % Your university, school and/or department name(s)
\horrule{0.5pt} \\[0.4cm] % Thin top horizontal rule
\huge Report on the Product Design Assignment \\ % The assignment title
\horrule{2pt} \\[0.5cm] % Thick bottom horizontal rule
}

\author{Joseph Holland} % Your name

\date{\normalsize\today} % Today's date or a custom date

\begin{document}

\maketitle % Print the title

%----------------------------------------------------------------------------------------
%	INTRODUCTION
%----------------------------------------------------------------------------------------

    \section{Introduction}

        The task was set to identify a specific problem with a mode of public transport and to devise a way to improve this situation by designing or redesigning a product, system or service.\\

        The problem that was identified was relating to the fact that commuters will often miss their train due to not leaving to walk to the station on time as they do not correctly judge how long it will take to get there.\\

        The solution was to design and build an android app that would determine the train that the commuter would catch and then notify the commuter of an appropriate time to leave their location in order to arrive at the station in time for their train.\\

        An additional feature that was developed into the application was a live map display of the current locations of trains in the UK.

%---------------------------------------------------------------------------------------
%	DESIGN
%---------------------------------------------------------------------------------------

    \pagebreak
    \section{Problem Identification}

        The initial task in the design process was to identify a suitable problem with a mode of public transport. To accomplish this, the problem space was widened and abstracted in order to allow for a range of problems to be made present. Below are listed a number of problems that were considered to be solved:

        \begin{itemize}
            \item Train related problems
            \begin{itemize}
                \item Train stops at many stops before your required stop, even if no passengers enter/leave the train

                \item Many journeys have the same fare, despite them being different distances

                \item Trains can only take you to locations where a station is available

                \item Some trains run with very few passengers

                \item Other trains run over capacity

                \item Trains run slowly around tight corners

                \item Many commuters miss their train and must either catch a later train or forfeit their fare

                \item Tickets are not always checked on the train for validity

                \item Trains require a large number of staff to operate
            \end{itemize}

            \item Bus related problems
            \begin{itemize}
                \item Buses experience traffic

                \item Buses that have a low demand have very high fares
            \end{itemize}
        \end{itemize}

    \section{Task Clarification}

        The problem that was selected was that commuters often miss their trains and thus will be late to arrive at their destination and can sometimes have to pay more for a later train.\\

        The need for a solution to this problem is great as it causes for the trains that are missed to be under capacity and for the trains that are caught late are over capacity. It also means that commuters can be late to arrive to their destination and spend additional money on fares unnecersarily.\\

        This problem was abstracted in order to cause for the widest range of possible solutions to be discovered and explored. After this abstraction process, the problem statement was as follows:\\
        \begin{center}
            \textit{Devise a way to reduce the number of people who miss their train during a commute}\\
        \end{center}

        Then, in order to realise the solution, a requirements list was drawn up:\\

        [Key: D/W = Demand/Wish; Wt = Weight (wish importance) between 1 and 3]

        \begin{center}
        \begin{tabular}{|| c | c || p{12cm} || c ||}
            \hline
            \textbf{D/W} & \textbf{Wt} & \textbf{Requirements} & \textbf{Keyword} \\
            \hline
            &&&\\
            && \textbf{Functionality} & \\
            \hline
            D && The system must determine train operating times & Timings \\
            D && The system must determine live train locations & Locations \\
            D && The system must determine User location and destination &  Directions \\
            D && The system must determine required train for journey & Journey \\
            D && The system must calculate an appropriate time for the user to travel to station & Leaving \\

            W & 3 & The system sets an alarm for when the user should begin journey & Alarm \\
            W & 2 & The user can select trains on live map and get details of selected train & Details \\
            W & 2 & The user can review directions for their selected journey & Review \\
            &&&\\

            && \textbf{Efficiency} & \\
            \hline
            D && The system must minimise calls to data APIs thus minimising required bandwidth & Bandwidth \\
            D && The system must minimise calculations on device & Battery \\
            W & 3 & The device should store minimal information & Size \\
            W & 2 & The device should make minimal API calls that are general/uniform & API \\
            &&&\\

            && \textbf{Useablilty} & \\
            \hline
            D && The application must have minimal bugs so as to minimise crashes & Bugs \\
            W & 3 & The user should make as few interactions as possible to recieve the required response & Interaction \\
            W & 2 & The system should be user-friendly and clear to navigate & Navigation \\
            W & 1 & The system should be intergrated with other systems to provide a seamless operation experience (e.g Google directions/maps intergration) & Integration \\
            &&&\\

            && \textbf{Aesthetics} & \\
            \hline
            W & 2 & The application should have a clean and user-friendly design & Design \\
            W & 1 & The application design should follow Google Material Design requirements & Material \\
            &&& \\

            && \textbf{Timescales} & \\
            \hline
            D && Product and work deadline: 25 April 2017 & Work \\
            D && Presentation date: 26 April 2017 & Presentation \\
            \hline

        \end{tabular}
        \end{center}

    \section{Conceptual Design}

        In order to begin designing a concept to solve the problem, the overall function of the solution must be determined. The overall function of the solution must be to provide the user with useful information and notification of train locations with respect to them in order for them to have the best chance of avoiding missing their train.\\

        This overall function was then decomposed into smaller sub-functions as detailed below:\\

        \begin{center}
        \begin{forest}
            for tree={
                align=center,
                parent anchor=south,
                child anchor=north,
                font=\sffamily,
                l sep+=10pt,
                edge path={
                    \noexpand\path [draw, \forestoption{edge}] (!u.parent anchor) -- +(0,-10pt) -| (.child anchor)\forestoption{edge label};
                },{}
            }
            [
                Present\\information
                [
                    Determine\\Train locations
                    [
                        Aquire\\raw train data
                    ]
                ]
                [
                    Determine\\train times
                    [
                        Gather\\directions data
                    ]
                    [
                        Gather\\timetables
                    ]
                ]
                [
                    Calculate\\journey times
                    [
                        Collect\\arrival/leaving times
                    ]
                    [
                        Gather\\timetables
                    ]
                ]
            ]
        \end{forest}
        \end{center}

        Next, solution priciples were determined and combinations of these principles were identified:

        \begin{center}
        \begin{table}[!ht]
        \small
        \begin{tabular}{|| c || m{3cm} | c | c | c ||}
            \hline
            \textbf{Function} & \multicolumn{4}{| c |}{\textbf{Solution Priciples}} \\
            \hline
            Present information & Graphically & As Text & On a map layout & Spoken \\
            Aquire raw train data & From GPS devices on trains & From public APIs && \\
            Gather Directions data & Design bespoke directions system & Use Google directions API&& \\
            Gather Timetables & Use Public Rail API&&&\\
            Collect Arrival/Leaving Times & Request user input&&&\\
            Determine Train Locations & Translate GPS data into latitude longitude & Analyse data from API&&\\
            \hline
        \end{tabular}
        \end{table}
        \end{center}


        \pagebreak
        Three combinations concepts were identified:

        \begin{center}
        \begin{table}[!ht]
        \small
        \begin{tabular}{|| c || m{3cm} | m{3cm} | m{3cm} ||}
            \hline
            \textbf{Function} & \multicolumn{3}{| c |}{\textbf{Combnation}} \\
            \hline
            & Combination 1 & Combination 2 & Combination 3\\
            \hline
            Present information & As Text & On a map layout & Spoken \\
            Aquire raw train data & From public APIs & From public APIs & From GPS devices on trains \\
            Gather Directions data & Use Google directions API & Use Google directions API & Design bespoke directions system  \\
            Gather Timetables & Use Public Rail API & Use Public Rail API & Use Public Rail API\\
            Collect Arrival/Leaving Times & Request user input & Request user input & Request user input\\
            Determine Train Locations & Analyse data from API & Analyse data from API & Translate GPS data into latitude longitude \\
            \hline
        \end{tabular}
        \end{table}
        \end{center}

        To identify the best solution, a concept evaluation was set up:

        \begin{center}
        \begin{table}[!ht]
        \small
        \begin{tabular}{|| l | c | c | c | c | c | c | c ||}
        \hline
        &&\multicolumn{2}{| c |}{Concept 1} & \multicolumn{2}{| c |}{Concept 2} & \multicolumn{2}{| c ||}{Concept 3}\\
        \hline
        Criteria & Wt. & value & score & value & score & value & score\\
        \hline
        Alarm & 3 & - & - & 0 & 0 & +2 & +6\\
        Details & 2 & - & - & +2 & +4 & -2 & -4\\
        Review & 2 & - & - & +1 & +2 & -1 & - 2\\
        Size & 3 & - & - & -2 & -6 & -2 & -6\\
        API & 2 & - & - & -1 & -2 & 0 & 0\\
        Interaction & 3 & - & - & +2 & +6 & +2 & +6\\
        Navigation & 2 & - & - & +1 & +2 & +2 & +4\\
        Integration & 1 & - & - & +1 & +1 & -2 & -2\\
        Design & 2 & - & - & +2 & +4 & +2 & +4\\
        Material & 1 & - & - & 0 & 0 & 0 & 0\\
        \hline
        \multicolumn{2}{|| r |}{total score} && 0 && +11 && +6\\
        \hline

        \end{tabular}
        \end{table}
        \end{center}

        Thus, the chosen concept was concept 2.

    \pagebreak
    \section{Embodiment Design}

        \subsection{Redundancy}

            This system will be built to function under the circumstances of sub-system failure. It will be capable of this by causing the sub-systems to be built upon the primary functionality. That primary functionality will be the system's ability to aquire appropriate direction data from the Google directions API. Any sub-system failure, such as the map view or the live train data capture, will not cause the directions API to fail. This will be achieved by coding the sub-systems as proprietary, with the base function able to operate without the sub-systems being available.

        \subsection{Tolerance}

            This system will be highly tolerant to unexpected or uncertain situations by the implementation of appropriate input analysis. This will revolve around a well structured user and API data input flow that is based on the principals of GIGO (Garbage In Garbage Out). This mean that the input flow will be properly designed to identify usefull and properly structured data coming in and to disregaurd data that is innapropriate. If the user input is innapropriate, the system will warn the user of the fact and will require the input to be repeated until the data is acceptable. If the innapropriate data comes from the API, the system will simply disreguard the data and will continue to request the data from the API until the correct data is received.

        \subsection{Safety}

            This system will be entirely safe as it is not a physical product. The primary safety concerns surrounding the use of this product are covered by the safety systems of the hardware being used to run it. Other than this, the user may pay attention to the appllication over the surroundings of the user, putting them in potentially hazourdous situations. This will be minimised by providing the user with appropriate warnings of this fact and to caution them to keep their attention in the proper places.

    \section{Design for People}

        \subsection{Stakeholder Analysis}

            Production - This application would be produced by a team of programmers, engineers and designers. To streamline the process for this team, appropriate workloads and deadlines must be applied to them in order to reduce their chances of over stressing.\\

            Distribution - This application would be distributed by the appropriate app stores and thus, once the application is uploaded, no human interaction is involved.\\

            Installation - This application is installed by the user, to minimise inconvinience, the application should be properly optimised to reduce download and installation time.\\

            Use - This application must be optimised and designed properly to simplify the user experience. This can be accomplished by minimising the number of button presses the user must execute in order to accomplish any given task. In addition to this, clear labelling of buttons and other points of interest with clear text or with simple graphics can be utilised.\\

            Maintenance - The programmers that produced the code for this application will also have to maintain it by removing any bugs that surface during the product lifetime. To make this process as simple as possible, code will have to be clearly commented. This will include breif descriptions of how functions operate and the purpose of classes and other system modules.

        \subsection{Diversity Analysis}

            To allow this product to be accesible to a diverse population, the following considerations have been taken into account:

            \begin{itemize}
                \item The product should be adapted to offer a variety of languages, to alllow access to those whose first language is not english

                \item The product should be built to work in many countries to allow people to access the product for a wide variety of places

                \item The product should be as simple to use as possible to allow the use by a wide range of ages and learning ablility

                \item The product should include large buttons and interfaces to allow access by those with impared vision
            \end{itemize}

        \pagebreak
        \subsection{Task Analysis}

            Below is detailed the operation flow of the application:

            \pagebreak
            \subsubsection{sub-task analyis}
                \begin{itemize}
                    \item Input of locations
                        \begin{itemize}
                            \item The user must be able to easily navigate to the desired locations. This can be achieved by utilising the simple touch gestures offered by the Google Maps API
                            \item The user can also enter this information though a text interface, this gives the user the chance to enter the data by their prefered method
                        \end{itemize}

                    \item User gets further information on their desired route
                        \begin{itemize}
                            \item This is simplified for the user as it is achieved by simply performing a click operation on the displayed route
                        \end{itemize}

                    \item User alarm
                        \begin{itemize}
                            \item This operation requires the user to check the option for an alarm to be set in the application options
                            \item This operation makes the application more accesible as it will warn the user to begin their journey without any further user cognition
                        \end{itemize}
                \end{itemize}




\end{document}
%
